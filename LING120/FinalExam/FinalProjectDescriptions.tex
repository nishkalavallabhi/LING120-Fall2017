\documentclass[11pt,a4paper]{article}
\usepackage{hyperref}

\begin{document}
\begin{center}
  Fall Semester 2017 \\ Iowa State University\\[3ex]
  {\large ENGL 120 - Language and Computers}\\[3ex]
  \textbf{Final Project Descriptions} \\ \textbf{Submission Deadlines: \\
  Part 1: 2nd December 2017
  Part 2: 6th December 2017
  Part 3: 13th December 2017}
\end{center}

\paragraph{Instructions:} Your final exam involves writing a technical report and is submitted in three parts:
\begin{itemize}
\item Submit a preliminary draft (150-200 words) for Question 1 (5 points - due on 2nd December)
\item Do peer-review of 2 of your classmates and give them constructive feedback to improve the drafts (5 points - due on 6th December, IN THE CLASS)
\item Submit a final report (10 points - due on 13th December)
There are two questions.For Question 1, you have to choose from a list of given topics. You can choose a topic beyond these, but you need to get my approval before Thanksgiving break. Question 2 is common for all. 
\end{itemize}

\section{Topics for Question 1}
Described below are some topics for Question 1. You can choose any one of them. For all these topics, you are required to spend some time looking for relevant reading materials online. Cite any materials you found useful for preparing your writeup at the end of.

\paragraph{Text Input: } Present a study of different ways of typing a text in Chinese/Hebrew/Russian (or any such non-Roman script) on your computer or mobile phone. Mention atleast two ways of doing this, and compare them in terms of some criteria you can come up with. 

\paragraph{Machine Translation: } Compare google translate and bing translate for any language pair you want. Come up with some set of criteria and write an evaluation of both. 
Something about Eliza and the likes? 

\paragraph{Search: } Compare any two existing search engines (google search vs bing search) in terms of their general language understanding capabilities and working with multiple languages.

\paragraph{Google n-gram viewer: } Read about google n-gram viewer. Access the online demo, and think about some potential uses of this tool. How is it created? What are the potential problems you foresee with using this tool for literary analysis? Are there any existing analyses of this tool you found? 

\paragraph{Speech Synthesis: } lyrebird.ai promises to synthesize your voice after recording 30 sentences of you speaking. Create a login and evaluate this tool. 

\paragraph{Speech Recognition: } Evaluate the dictation tool in google docs for any language of your choice.


\paragraph{The following topics require Computer Science background: }

\paragraph{Language Generation:} Read the following article, and write a summary of what Zero-shot translation is all about, and why it works. \\ 
\url{https://research.googleblog.com/2016/11/zero-shot-translation-with-googles.html} 

\paragraph{Coding a chatbot: } \url{https://wit.ai/getting-started} -check this out, and write a summary of your experiences with it, and effort involved in writing code for creating chat bots.

\paragraph{Google/MS APIs: } google cloud API or MS cognitive services - use free versions, write a small program that can work with this. I don't want to run your programs - just show me screenshots of how it works. 

\section{Question 2}
Describe the impact of language technologies on the society. Compare and contrast your interactions with language technologies 5 years back and now, and write how you think our daily life will be impacted by language technologies in the coming 5--10 years. 
(Question 2 is common and mandatory for everyone.)

\section{Instructions for preliminary draft submission (Part 1)}
Your report can be 150-200 words long. This report should contain details about your chosen topic for Question 1, and what issues related to this topic do you want to cover in the final exam submission. You should submit this as a PDF file in the Canvas Assignments submission, AS WELL AS in the forum titled "FINALS-Part1" i.e., your submissions can be seen by the whole class.

\section{Instructions for peer review (Part 2)}
This will be done in the class on 6th December. Each student must read one other person's submission and give feedback on what they wrote.  Feedback can include issues such as: some suggestions about examples to include, or criteria to evaluate, languages to compare etc. You are not obliged to address all these reviewer comments in your final submissions. These are just suggestions.

\section{Instructions for final submission (Part 3)}
Submit a single PDF file, with two questions - Write the question before you start your answer (for Q1, please don't list all options. Only put the question that you chose!). Answer for each question can be 350-400 words long. For both questions, answer clearly, and without vagueness. Whether you are evaluating a specific application or the general impact of language technology on the society, define a set of criteria for your evaluation (2--4) and evaluate according those criteria.  

\end{document} 


