\documentclass{beamer}
\usepackage[utf8]{inputenc}

\author[Sowmya Vajjala]{Instructor: Sowmya Vajjala}


\title[LING 120]{LING 120: Language and Computers}
\subtitle{Semester: FALL 2017}

\date{1 Sep 2017}

\institute{Iowa State University, USA}

%%%%%%%%%%%%%%%%%%%%%%%%%%%

\begin{document}

\begin{frame}\titlepage
\end{frame}

\begin{frame}
\frametitle{Class outline}
\begin{itemize}
\item Clarification about terms used yesterday
\item A group exercise preparing you for next week
\end{itemize} 
\end{frame}

\begin{frame}
\frametitle{Clarification about terms used on Wednesday}
\begin{itemize}
\item Non-word error detection
\item Isolated word error correction
\item Contexual spelling correction and grammatical error correction (for real-word errors)
\end{itemize} 
\end{frame}

\begin{frame}
\frametitle{Non-word error detection - Dictionaries}
\begin{itemize}
\item Dictionary based - but it does not cover all cases, as we saw in the last class.
\item Usual solution: maintain a dictionary with different word forms etc (add some exceptions - like ignore words with caps etc.) .. 

\item Most spell-checking tools support personalized spell-checking. (What does that mean??)
\end{itemize}
\end{frame}

\begin{frame}
\frametitle{Non-word error detection - N-gram methods}
\begin{itemize}
\item To some extent, we can detect misspellings even without a dictionary. 
\item Before talking about "how", what is a character n-gram? \pause
\item If we collect a large list of valid English words, we can have an estimate of character sequence frequencies in English. 
\item We can then use it to check for abnormal character sequences in a word, to identify spelling errors. (Makes sense??) \pause
\item example: "thi" is perhaps a valid character trigram in English, but "qki" is not. 
\end{itemize}
\end{frame}

\begin{frame}
\frametitle{Isolated word spelling correction}
\begin{itemize}
\item Aim: suggest correction candidates for a mis-spelled word, irrespective of the surrounding words.
\item Question to test if you did the reading: What are the three steps to do isolated word spelling correction? \pause
\begin{itemize}
\item Detect a error (dictionary or n-gram method)
\item Identify possible candidates for right spelling (from where??)
\item Rank them in terms of the most probable word (how?)
\end{itemize}
\end{itemize}
\end{frame}

\begin{frame}
\frametitle{Group Exercise}
\begin{itemize}
\item Form into groups of 2--3 people, and do the exercise described in the given sheet.
\item Work together, perhaps spend about 20-25 minutes on the problem, and we will use the remaining time for discussion of what you did.
\item This counts as your attendance for today.
\end{itemize}
\end{frame}

\begin{frame}
\frametitle{Background Information}
\begin{itemize}
\item n-grams are sequences of n-words.
\item In the sentence: "I have a book", "I" is a word unigram/1-gram. "I have" is a word bi-gram/2-gram "I have a" is a word trigram/3-gram. "I have a book" is a word 4-gram. and so on.
\item "I hav" is a character 5 gram with 5 characters - I, space, h, a, v
\item If I have a large collection of text files (called \textbf{corpus}), I can compile large dictionaries of such word and character n-grams of any arbitrary n, and how frequently are they seen in this corpus.
\item An example of such a large text corpus is: Wikipedia
\end{itemize}
\end{frame}

\begin{frame}
\frametitle{Next Week}
\begin{itemize}
\item Topic 1: Isolated spelling error correction
\item Topic 2: Contextual spelling error correction
\item Deadlines: Assignment 1 deadline on September 8th.
\item Reminder: No class on Monday.
\item Readings: Finish reading Chapter 2
\end{itemize}
\end{frame}


\end{document}

\begin{frame}
\frametitle{Generating possible candidates}
\begin{itemize}
\item e.g., SOUNDEX
\item e.g., edit distance 
\item e.g., Character transition probabilities
\end{itemize} %one slide on each?
\end{frame}

\begin{frame}
\frametitle{Next Week ..} 
\begin{itemize}
\item Topic: Grammar correction
\item Remember: Assignment 1 is due on September 8th next week
\item Readings: Read Section 2.4 in Chapter 2 in the textbook.
\item To Do: Work on your assignment!
\end{itemize}
\end{frame}
%advnced: under the hood - dynamic programing stuff.

%exercise: From what you saw so far, which one may be more effective - Soundex or edit distance or character transition probs?


