\documentclass{beamer}
\usepackage[utf8]{inputenc}

\author[Sowmya Vajjala]{Instructor: Sowmya Vajjala}


\title[LING 120]{LING 120: Language and Computers}
\subtitle{Semester: FALL 2017}

\date{28 Aug 2017}

\institute{Iowa State University, USA}

%%%%%%%%%%%%%%%%%%%%%%%%%%%

\begin{document}

\begin{frame}\titlepage
\end{frame}

\begin{frame}
\frametitle{Class Outline}
\begin{itemize}
\item Quick recap of last week %5-10min
\item Encoding speech  - continued
\item Relating written and spoken language - an overview
\item Wrapping up Topic 1! %talk here about language input
\end{itemize}
\end{frame}

\begin{frame}
\frametitle{Encoding Text: Recap from last week}
\begin{itemize}
\item ASCII - 7 bit encoding
\item Extending ASCII to accomodate new languages
\item Unicode
\item 3 different ways of Unicode conversion: UTF8, UTF16, UTF32
\item How different writing systems look different on browser if you change the encoding. 
\end{itemize}
\end{frame}

\begin{frame}
\frametitle{Review question on encoding text}
What is "America" (English letters) in UTF-8, UTF-16, UTF-32 - will it be same, or similar, or totally different? \pause
\medskip \\ Useful website (you can also use this for Assignment 1 instead of the other one): \url{http://macchiato.com/unicode/convert.html}
\end{frame}

\begin{frame}
\frametitle{Encoding Speech: Recap from last week}
What speech properties are interesting?
\begin{itemize}
\item speech rate (fluency, number of pauses etc)
\item Loudness/amplitude 
\item What sound frequencies correspond to different characters in human speech? 
\item How can we tell sounds apart with this frequency information?
\item Pitch - how high or low is a sound (useful especially for identifying vowels)
\item Intonation - rise and fall of pitch
\end{itemize}
\end{frame}

\begin{frame}
\frametitle{Question from Friday's class}
The following phrases/sentences represent some mishearings of songs and possible errors that a speech recognition software can also make. Try to guess an alternate version and post your responses on Canvas forum for today. That is your attendance for today:
\begin{itemize}
\item Example: "How to wreck a nice beach" - "How to recognise speech"
\item "Secret agent man" %secret asian man
\item "when the rainbow shaves you clean, you'll know" %when the rain washes you clean you'll know
\item "with my knee on my mind" %with money on my mind
\item "language interpreters" %language and computers
\item "synthetic meditation" %syntactic annotation 
\end{itemize}
Note: People who did not answer did not get attendance for friday. 
\end{frame}

\begin{frame}
\frametitle{Answers}
\begin{itemize}
\item "Secret agent man" - secret asian man
\item "when the rainbow shaves you clean, you'll know" - when the rain washes you clean you'll know
\item "with my knee on my mind" - with money on my mind
\item "language interpreters" - language and computers
\item "synthetic meditation" - syntactic annotation 
\end{itemize}
(First 3 are popular Mondegreens, last 2 are speech recogniser output for my speech).
\end{frame}

\begin{frame}
\frametitle{Some of your responses from Friday}
\begin{itemize}
\item Will you grab my peas - heard as "will you grab my keys"
\item "grey chair" for "great share"
\item "The slicer for the next class" - "The slides for phonetics class"
\item ”language interpreters” - languish under orders, or vintage interloper?
"Synthetic meditation." - "Should take medication."/"send the invitation"
\end{itemize}
note: Some popular Mondegreens can be seen at: \url{http://www.uh.edu/~mbarber/mondegreens.html}. You can see more on youtube.
\end{frame}

\begin{frame}
\frametitle{How do we understand speech signals on a computer?}
\begin{itemize}
\item Oscillogram: shows time on X-axis and changes in signal amplitude on y-axis 
\item Spectrogram: shows frequency on X-axis and time on Y-axis
\item Darkness of a spectrogram: loudness of a sound.
\item Sound frequencies, change of darkness, (and several other such measures) help us measure speech and understand words in the speech. \pause
\item Let us see an example using one software that analyses speech - Praat \pause 
\end{itemize}
Note: Read "Under the Hood 1" in Chapter 1 if you are curious about Spectrograms. Some of it is a part of Assignment 1 too!
\end{frame}

\begin{frame}
\frametitle{Praat}
\begin{itemize}
\item Praat is a free software package to analyse speech signals.
\item You can record sounds, visualize the spectrogram and oscillograms for these sounds with Praat.
\item You can pull out a smaller section of these files for further analysis
\item You can alter the speech files and create new files
\item You can even measure things like: creakiness, nasality in voice etc!
\end{itemize}
More information: \url{http://savethevowels.org/praat/UsingPraatforLinguisticResearchLatest.pdf}
\end{frame}

\begin{frame}
\frametitle{Quick Demo of Praat}
How do three ways of saying the same word differ in their spectrogram images? \pause \\ 
... You will need to work with this tool for your Assignment 1!
\end{frame}

\begin{frame}
\frametitle{Relationship between written and spoken language}
\begin{itemize}
\item Automatic speech recognition: converting a speech sample into text representation
\item Text to speech: converting text into speech representation
\end{itemize}
\pause According to you, which is easier? Why?
\end{frame}

\begin{frame}
\frametitle{Automatic Speech Recognition}
Some major issues in achieving this: 
\begin{itemize}
\item Converting those recordings into individual sounds, and words.
\item Identifying word boundaries (remember: there are no punctuation markers or white spaces as in text!)
\item Different people pronounce differently - different accents.
\item The issue of multiple languages occurs here too!
\end{itemize}
More on this when we talk about dialog systems. 
\end{frame}


\begin{frame}
\frametitle{Text to Speech Synthesis}
Seems like a trivial task?
\begin{itemize}
\item Isn't it just recording individual sounds (called phones) and stitching them together by looking at text and form speech automatically??? \pause
\item Yes, but different phones also sound differently based on their neighboring sounds - we need context for right pronunciation of a word and its phones! \pause
\item Further, the synthesized sound should feel human, not look robotic! 
\item Question: how many of you use Waze? Did you try changing voices of the speakers?
\end{itemize}
\end{frame}

\begin{frame}
\frametitle{Writing Text on computer}
We saw how to render a text on a computer (using encodings) so that you can read. We also saw a little bit about how to represent speech. What about writing?
\pause \\ Three common methods exist: 
\begin{itemize}
\item Using keyboard layouts
\item Using soft keyboards
\item Using phonetic mappings
\end{itemize}
Useful link: \url{https://en.wikipedia.org/wiki/Input_method}
\end{frame}

\begin{frame}
\frametitle{Next Class}
\begin{itemize}
\item Topic 2: Writing aids - introduction %5-10min
\end{itemize}
\end{frame}

\begin{frame}
\frametitle{Attendance for Today}
If I record you speak, use ASR and convert it to text, and then use TTS to convert it again to speech - do you think I will get your speech sample back? Write a short explanation for your answer (few sentences). You can either write it now and give it, or post on Canvas forum for today and get attendance.
\end{frame}

\end{document}

Start monday talking about mondegreens:
%Exercise: "How to wreck a nice beach" vs "How to recognise speech", vs "How to wreck a nice beach"

%Question for Monday: mention one of the problems of speech processing (no breakup of words) and ask how they will solve it. 

%continuation on friday: \item How do we type them on a computer? %10min
%and then to speech.
%friday exercise: how do i type Telugu on MacOS?
%Assignment 1 description

%question for next monday: %\item To Do (optional): Figure out how to type German specific characters on your computer (umlauts)
