\documentclass{beamer}
\usepackage[utf8]{inputenc}
\usepackage{graphicx}
\usepackage{url}

\author[Sowmya Vajjala]{Instructor: Sowmya Vajjala}

\title[LING 120]{LING 120: \\ Language and Computers}
\subtitle{Semester: Fall '17}

\date{1 December 2017}

\institute{Iowa State University, USA}

%%%%%%%%%%%%%%%%%%%%%%%%%%%

\begin{document}

\begin{frame}\titlepage
\end{frame}

%Next class: Evaluation of MT, Conclusion of this topic. Remind of Finals 1.

\begin{frame}
\frametitle{Today's class}
\begin{itemize}
\item Reminder about Finals and the related deadlines
\item Discussion about questions from last class
\item Evaluation of MT
\item Conclusion of MT
\end{itemize}
\end{frame}

\begin{frame}
\frametitle{Discussion about question from last class}
\begin{itemize}
\item Task: identify English-Indonesian word translations based on the provided comparable article between the two languages.
\item Solution: \url{http://nacloweb.org/resources/problems/2011/FS.pdf}
\item Strategy: Not looking for exact meanings (we don't know the other language!). Looking for contexts (surrounding words), frequency of occurrence, words that go together etc.
\end{itemize}
\end{frame}

\begin{frame}
\frametitle{Discussion about question from last class}
Different ways of machine translation:
\begin{itemize}
\item Example based MT: Look for matching words or phrases in previous human translations, and assist human translators by doing this. 
\item Direct MT: Word by word generation of translation
\item Syntactic Transformation based MT: Rules for changing syntactic structure between source and target languages
\item Combining Direct and Syntactic MT
\item Statistical MT (learning from data)
\end{itemize}
\end{frame}

\begin{frame}
\frametitle{Questions before I start with evaluation}
\begin{itemize}
\item What is a good way to evaluate MT? \pause
\item Is doing human evaluations easy or difficult?
\item If I don't have humans sitting there to do evaluations, how can I do automatic evaluations? \pause
\item What aspects should be considered while doing evaluation of MT systems? \pause
\item Why bother about evaluation, really? \pause
\item How are human translations evaluated in professional translators' world? \pause
\end{itemize}
\end{frame}

\begin{frame}
\frametitle{Going backwards with answers-1}
\begin{itemize}
\item Q: How are human translations evaluated in professional translators' world?
\item A: Quality is evaluated in terms of number of words edited/corrected in post-translation editing and proof reading.
\item Q: Can we apply the same measure to MT? \pause
\item A: May be not sufficient. First, MT may always need editing, second: writing patters of a human and a machine can be completely different. We may need more evaluation measures along with this.
\end{itemize}
\end{frame}

\begin{frame}
\frametitle{Going backwards with answers-2}
\begin{itemize}
\item Q: Why bother about evaluation, really?
\item A: Some example scenarios: 
\begin{itemize}
\item For end users: should I really use MT on my website? Is it any good, or will it reduce my customer base because it will frustrate them? \pause
\item For MT software developers: How do I know how well are new changes in my program improving the translation quality? \pause
\item For researchers: How do we compare different MT systems around the world, made for different reasons, by different people, following different approaches? \pause
\end{itemize}
\end{itemize}
\end{frame}

\begin{frame}
\frametitle{Going backwards with answers-3}
\begin{itemize}
\item Q: What aspects should be considered while doing evaluation of MT systems?
\item A: many
\begin{itemize}
\item Quality (naturalness, fluency) of translations at sentence level, at the level of full text, quality related to task at hand
\item Accuracy of translations (no missing info, no additional info)
\item What kind of errors does the system make, what kind of errors it doesn't make
\item How does it work with different genres of writing? (fine grained error analysis)
\item Is it possible to tag "bad translations" to be sent to human translators?
\item How much of post-editing will humans have with this output? 
\end{itemize}
... and so on.
\end{itemize}
\end{frame}

\begin{frame}
\frametitle{Going backwards with answers-4}
\begin{itemize}
\item Q: If I don't have humans sitting there to do evaluations, how can I do automatic evaluations?
\item A: Ask humans to translate some sentences, give those sentences to MT systems, and compare the degree of overlap between human and machine translations.
\item There are different evaluation measures that focus on doing this. 
\item Advantage: automatic evaluation, so we don't have to conduct expensive, time-consuming evaluation studies anymore.
\item Disadvantage? \pause - we don't know if the MT system is really good or just good on those sentences. We also don't know what is the best measure of closeness to human translation!
\end{itemize}
\end{frame}

\begin{frame}
\frametitle{Going backwards with answers-5}
\begin{itemize}
\item Q: Is doing human evaluations easy or difficult?
\item A: Time and Cost. Reliability and Consistency. Meaningfully summarizing human judgements. 
\item Time and cost: can be addressed by doing crowdsourcing instead of looking for expert human translator judgements. 
\item Other issues: Difficult
\end{itemize}
\end{frame}

\begin{frame}
\frametitle{Going backwards with answers-6}
\begin{itemize}
\item Q: What is a good way to evaluate MT?
\item A: Who is John Galt?
\end{itemize}
\end{frame}

\begin{frame}
\frametitle{Why is it difficult to evaluate MT?}
\begin{itemize}
\item Language variability - there is no single correct translation!
\item Human evaluation is subjective
\item For what purpose are we using MT?
\item How do we evaluate evaluation?
\item How do we decide how good is "good enough"?
\end{itemize}
Summary: MT evaluation is a research topic in itself! :)

Reference: Slides by Alon Lavie, MT Summit, 2011
\end{frame}

\begin{frame}
\frametitle{State of MT today}
\begin{itemize}
\item We have usable applications that work across many language pairs (Google Translate, Bing Translate, Babel, Facebook's translate etc)
\item Not only are they used in different devices, they can also be deployed on your own websites, or applications as a service (if you know some programming, and have some money to license these)
\\ e.g.: \url{https://azure.microsoft.com/en-us/pricing/details/cognitive-services/translator-text-api/}
\\ e.g., \url{https://cloud.google.com/translate/pricing}
\item They are still far from perfect, but are actively improving.
\item One application that can have a potential impact on multiple aspects of human society in decades to come (something to think about for writing Q2 in Final Exam!)
\end{itemize}
\end{frame}

\begin{frame}
\frametitle{Next Week}
\begin{itemize}
\item Theme: Revision and some group exercises in the class
\item Deadlines:
\begin{itemize}
\item Final Exam, part 1: Writing a short summary of your Q1 topic (Due tomorrow)
\item Final Exam, part 2: Do a peer review for one person's Q1 draft above (IN THE CLASS, on 6th December. Don't miss the class!)
\item Final Exam, part 3: Submit Q1 and Q2, final versions in exams week (13th DEC)
\end{itemize}
\item NOTE: No extensions on any of these deadlines!
\end{itemize}
\end{frame}

\begin{frame}
\frametitle{Attendance Question}
\begin{itemize}
\item Background:
\begin{itemize}
\item There is this online competition called "Spooky Author Identification" \\ (\url{https://www.kaggle.com/c/spooky-author-identification})
\item Task: you're challenged to predict the author of excerpts from horror stories by Edgar Allan Poe, Mary Shelley, and HP Lovecraft.
\item How does it work: First, they release some set of texts where authors are known, and are one of these.
\item You should then do "something" which can predict the author of a new text (who among these 3 is most likely?)
\end{itemize}
\item Question: What ideas from this course are useful in solving this problem, conceptually speaking? How?
\end{itemize}
\end{frame}

\end{document}

