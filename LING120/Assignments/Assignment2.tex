\documentclass[11pt,a4paper]{article}
\usepackage{url}

\begin{document}

\begin{center}
  Fall Semester 2017 \\ Iowa State University\\[3ex]
  \large LING 120 - Language and Computers\\[3ex]
  \textbf{Assignment 2} \\ \textbf{Submission Deadline: 23 SEP 2017, end of the day}
\end{center}

\paragraph{Instructions:} This assignment consists of two questions and carries a total of 10 marks. Submit your assignment as a *PDF* file and name it as: your\_first\_name--your\_last\_name.pdf Late submissions are allowed, but will not be awarded full credit.

\section*{Question 1 - 4 Marks} 
Listed below are some of the real spelling errors made by students taking a test, for a word \textit{learning}:\\
\textit{leaniong \\
learing \\
learnig \\
learnin \\
learnind \\
learniong \\
leerning \\}
Do you think there is any theoretical explanation for these errors? (e.g., getting confused with some other word, adjacent letters on a keyboard, etc.) or are they all just random errors? For many of these does your favorite spell checker (Word, Google, Grammarly etc) suggest "learning" as the correct spelling? What were the other words the spell checker suggested? 


\section*{Question 2 - 6 marks}
Consider the following examples from Figure 2.14 in the textbook (and a few more):\\
\textit{power crd \\
video crd \\
power curd \\
platnuin rings \\
golf war \\
sap opera \\
Saddam participated in golf war. \\
I want to buy platnuin rings. \\
I went their book.\\}

\begin{itemize}
\item Are all these errors of the same kind? What are the differences? 
\item What in your view is the most difficult error to catch among these?
\item Consider three spelling and grammar checking tools (Microsoft Word, Google search textbox, Grammarly.com) and check what suggestions do they offer for these errors. Which of them is doing a better job? Why do you think it is doing a better job? 
\end{itemize}



\end{document}
